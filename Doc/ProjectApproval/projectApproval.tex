\documentclass[12pt]{article}

\usepackage{graphicx}
\usepackage{paralist}
\usepackage{hyperref}
\hypersetup{
    colorlinks=true,
    linkcolor=blue,
    urlcolor=cyan,
}
\usepackage{xspace}
\usepackage{amsfonts}
\usepackage{amsmath}

\newcommand{\latex}{\LaTeX\xspace}

\oddsidemargin 0mm
\evensidemargin 0mm
\textwidth 160mm
\textheight 200mm
\renewcommand\baselinestretch{1.0}

\pagestyle {plain}
\pagenumbering{arabic}

\newcounter{stepnum}

\title{
    SFWRENG 3XA3 Project Approval\\
    \large Department of Computing and Software\\
}
\author{}
\date{\today}

\begin {document}

\maketitle

\section{Possible Project Ideas}
\begin{itemize}
    \item Connect Four
    \begin{itemize}
        \item Add difficulty levels by changing AI algorithm
        \item Add more than 2 players
    \end{itemize}
    \item Discord Meeting Bot
    \begin{itemize}
        \item Add scrum features (scrum master, sprints, stand-ups)
        \item Connectivity to Google Calendar
        \item Set deadlines
        \item Connectivity to Trello or Jira Boards
        \item Record meeting minutes
        \item Plan locations using Google Maps API
        \item 
    \end{itemize}
\end{itemize}


\section{Team Name and Members}

\begin{tabular}{l l l}
    Team Name: & BIGGIE SHAQ\\
    \\
    Arkin Modi & modia1 & 400142497 \\
    Leon So & sol4 & 400127468 \\
    Timothy Choy & choyt2 & 400135272
\end{tabular}

\section{Original Project Information}
\subsection{Project Description}
The original source of our project came from Code-Plus-Plus' Discord Meeting Bot. It is an open source Discord bot with some simple commands to run and plan meetings through Discord. The program is written in Javascript, and has the commands to add and display meetings and welcome new users. The bot schedules meetings to a date and time.\\
A Discord bot is a chat bot that runs on the platform Discord. There can be many uses for a Discord bot, from playing music to creating and planning meetings (what we are doing).

\subsection{URL of Original Project}
You can find the link to the original project \href{https://github.com/Code-Plus-Plus/discord-meeting-bot}{here}.

\section{Software Purpose and Scope}

\section{Specialized Hardware Requirements}
This project requires no additional specialized hardware.

\section{Licenses}
The project requires the MIT License, which allows for copying, modifying, merging, publishing, distributing without restriction. Therefore, we should be able to use this open source code as the basis for our project.

\section{Programming Language}

\section{Domain Knowledge}

\section{Test Cases of Existing Software}


\end {document}
