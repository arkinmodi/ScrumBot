\documentclass[12pt, titlepage]{article}

\usepackage{fullpage}
\usepackage[round]{natbib}
\usepackage{multirow}
\usepackage{booktabs}
\usepackage{tabularx}
\usepackage{hyperref}
\usepackage{graphicx}
\usepackage{float}
\hypersetup{
    colorlinks,
    citecolor=black,
    filecolor=black,
    linkcolor=red,
    urlcolor=blue
}
\usepackage[round]{natbib}

\newcounter{acnum}
\newcommand{\actheacnum}{AC\theacnum}
\newcommand{\acref}[1]{AC\ref{#1}}

\newcounter{ucnum}
\newcommand{\uctheucnum}{UC\theucnum}
\newcommand{\uref}[1]{UC\ref{#1}}

\newcounter{mnum}
\newcommand{\mthemnum}{M\themnum}
\newcommand{\mref}[1]{M\ref{#1}}

\title{SE 3XA3: Module Guide\\ScrumBot}

\author{
	Team 304, ScrumBot
		\\ Arkin Modi, modia1
        \\ Leon So, sol4
        \\ Timothy Choy, choyt2
}
\date{Last Updated: \today}

%\input{../../Comments}

\begin{document}

\maketitle

\pagenumbering{roman}
\tableofcontents
\listoftables
\listoffigures

\begin{table}[H]
    \caption{Revision History} \label{TblRevisionHistory}
    \begin{tabularx}{\textwidth}{llX}
        \toprule
            \textbf{Date} & \textbf{Developer(s)} & \textbf{Change}\\
        \midrule
            January 23, 2020 & Arkin Modi & Copy template\\
            March 5, 2020 & Leon So & Introduction, Anticipated Changes, Unlikely Changes, Module Hierarchy\\
            March 10, 2020 & Arkin Modi & Worked on Module Hierarchy\\
            March 11, 2020 & Arkin Modi & Worked on the Introduction\\
            March 12, 2020 & Arkin Modi & Worked on Introduction, Anticipated and Unlikely Changes, Module Hierarchy, Module Decomposition, Traceability Matrix and Use Hierarchy Between Modules\\
            March 12, 2020 & Leon So & Anticipated Changes and Unlikely Changes\\
            March 13, 2020 & Arkin Modi & Worked on Use Hierarchy Between Modules, Traceability Matrix, Module Decomposition, and Module Hierarchy\\
        \bottomrule
    \end{tabularx}
\end{table}

\newpage

\pagenumbering{arabic}

\section{Introduction}
\subsection{Overview} 
% Copied from Test Plan
Scrum is an Agile process framework widely used in industry for managing and coordinating collaborative projects. Scrum follows a highly iterative process and often has heavy customer involvement, therefore it can be often be complex. With Discord being a popular communication tool used by many teams of software developers today, ScrumBot provides a solution that directly integrates the management of a scrum development cycle into the communication channels. ScrumBot will allow for better management and organization of retrospectives, stand-ups, and other scrum/agile stages used by software teams within their routine communication channel. ScrumBot will provide features to add and manage Scrum meetings, as well as to store information relevant to those meetings. ScrumBot will also allow Scrum roles to be assigned to members of the Discord channel.

% ScrumBot is a Discord chat bot which allows software development teams to manage their Scrum agile software development process directly within the Discord application.\\

\subsection{Context}
Prior to this document, the Software Requirements Specification (SRS) was created to outline all the functional and non-functional requirements this project must satisfy. The purpose of this document is to provide a high-level structure to the implementation of this project by decomposing the idea into modules. The decomposition into modules enables a clearer form to satisfy the requirements of the project.\\

\noindent The Module Interface Specification was created in parallel to this document. It describes the operations that each module shall perform.\\

\noindent The Module Guide (MG) is developed~\citep{ParnasEtAl1984}. The MG specifies the modular structure of the system and is intended to allow both designers and maintainers to easily identify the parts of the software. The potential readers of this document are as follows:

\begin{itemize}
\item New project members: This document can be a guide for a new project member
  to easily understand the overall structure and quickly find the
  relevant modules they are searching for.
\item Maintainers: The hierarchical structure of the module guide improves the
  maintainers' understanding when they need to make changes to the system. It is
  important for a maintainer to update the relevant sections of the document
  after changes have been made.
\item Designers: Once the module guide has been written, it can be used to
  check for consistency, feasibility and flexibility. Designers can verify the
  system in various ways, such as consistency among modules, feasibility of the
  decomposition, and flexibility of the design.
\end{itemize}

\noindent The rest of the document is organized as follows. Section \ref{SecChange} lists the anticipated and unlikely changes of the software requirements. Section \ref{SecMH} summarizes the module decomposition that was constructed according to the likely changes. Section \ref{SecConnection} specifies the connections between the software requirements and the modules. Section \ref{SecMD} gives a detailed description of the modules. Section \ref{SecTM} includes two traceability matrices. One checks the completeness of the design against the requirements provided in the SRS. The other shows the relation between anticipated changes and the modules. Section \ref{SecUse} describes the use relation between modules.

\subsection{Design Principles}
% Copied from template
Decomposing a system into modules is a commonly accepted approach to developing
software.  The Module Guide (MG) developed by \citep{ParnasEtAl1984} specified the modular structure of the system. A module is a work assignment for a programmer or programming team~\citep{ParnasEtAl1984}.  Decomposition of the system into modules is
based on the principle of information hiding~\citep{Parnas1972a}.  This
principle supports design for change, because the ``secrets'' that each module
hides represent likely future changes.  Design for change is valuable in SC,
where modifications are frequent, especially during initial development as the
solution space is explored.\\

The design follows the rules layed out by \citet{ParnasEtAl1984}, as follows:
\begin{itemize}
\item System details that are likely to change independently should be the
  secrets of separate modules.
\item Each data structure is used in only one module.
\item Any other program that requires information stored in a module's data
  structures must obtain it by calling access programs belonging to that module.
\end{itemize}

\section{Anticipated and Unlikely Changes} \label{SecChange}

This section lists possible changes to the system. According to the likeliness
of the change, the possible changes are classified into two
categories. Anticipated changes are listed in Section \ref{SecAchange}, and
unlikely changes are listed in Section \ref{SecUchange}.

\subsection{Anticipated Changes} \label{SecAchange}

Anticipated changes are the source of the information that is to be hidden
inside the modules. Ideally, changing one of the anticipated changes will only
require changing the one module that hides the associated decision. The approach
adapted here is called design for
change.

\begin{description}
\item[\refstepcounter{acnum} \actheacnum \label{acInput1}:] The format of the input data
\item[\refstepcounter{acnum} \actheacnum \label{acInput2}:] The format of discord commands
\item[\refstepcounter{acnum} \actheacnum \label{acServer1}:] The URL used to connect to the hosting server
\item[\refstepcounter{acnum} \actheacnum \label{acServer2}:] The format of the response from the server
\item[\refstepcounter{acnum} \actheacnum \label{acOutput}:] The format of the output data
\end{description}

\subsection{Unlikely Changes} \label{SecUchange}

The module design should be as general as possible. However, a general system is
more complex. Sometimes this complexity is not necessary. Fixing some design
decisions at the system architecture stage can simplify the software design. If
these decision should later need to be changed, then many parts of the design
will potentially need to be modified. Hence, it is not intended that these
decisions will be changed.

\begin{description}
\item[\refstepcounter{ucnum} \uctheucnum \label{ucIO}:] 
\item[\refstepcounter{ucnum} \uctheucnum \label{ucInput}:] There will always be a source of input data external to the software
\item[\refstepcounter{ucnum} \uctheucnum \label{ucInput}:] The system will be interfacing with the Discord application
\item[\refstepcounter{ucnum} \uctheucnum \label{ucInput}:] The data structure of a meeting
\item[\refstepcounter{ucnum} \uctheucnum \label{ucInput}:] The data structure of a project
\item[\refstepcounter{ucnum} \uctheucnum \label{ucInput}:] The data structure of a task
\item[\refstepcounter{ucnum} \uctheucnum \label{ucInput}:] The data structure of a sprint
\end{description}

\section{Module Hierarchy} \label{SecMH}
This section provides an overview of the module design. Modules are summarized
in a hierarchy decomposed by secrets in Table \ref{TblMH}. The modules listed
below, which are leaves in the hierarchy tree, are the modules that will
actually be implemented.

\begin{description}
    \item [\refstepcounter{mnum} \mthemnum \label{m1}:] Hardware-Hiding Module
    \item [\refstepcounter{mnum} \mthemnum \label{m2}:] ScrumBot Module
    \item [\refstepcounter{mnum} \mthemnum \label{m3}:] Meeting Types Module
    % \item [\refstepcounter{mnum} \mthemnum \label{m?}:] Generic Dictionary Module
    \item [\refstepcounter{mnum} \mthemnum \label{m4}:] Meeting List Module
    \item [\refstepcounter{mnum} \mthemnum \label{m5}:] Meeting Module
    \item [\refstepcounter{mnum} \mthemnum \label{m6}:] Task List Module
    \item [\refstepcounter{mnum} \mthemnum \label{m7}:] Task Module
    \item [\refstepcounter{mnum} \mthemnum \label{m8}:] Sprint Module
    \item [\refstepcounter{mnum} \mthemnum \label{m9}:] Project List Module
    \item [\refstepcounter{mnum} \mthemnum \label{m10}:] Project Module
    \item [\refstepcounter{mnum} \mthemnum \label{m11}:] Generic Dictionary Module
\end{description}


\begin{table}[h!]
\centering
\begin{tabular}{p{0.3\textwidth} p{0.6\textwidth}}
\toprule
\textbf{Level 1} & \textbf{Level 2}\\
\midrule

{Hardware-Hiding Module} & ~ \\
\midrule

\multirow{1}{0.3\textwidth}{Behaviour-Hiding Module} 
& ScrumBot Module\\         % Interface with Discord
\midrule

\multirow{8}{0.3\textwidth}{Software Decision Module} 
& Meeting Types Module\\    % Data Structure
& Meeting List Module\\     % Data Structure
& Meeting Module\\          % Data Structure
& Task List Module\\        % Data Structure
& Task Module\\             % Data Structure
& Sprint Module\\           % Data Structure
& Project List Module\\     % Data Structure
& Project Module\\          % Data Structure
& Generic Dictionary Module\\   % Data Structure
\bottomrule

\end{tabular}
\caption{Module Hierarchy}
\label{TblMH}
\end{table}

\section{Connection Between Requirements and Design} \label{SecConnection}

The design of the system is intended to satisfy the requirements developed in
the SRS. In this stage, the system is decomposed into modules. The connection
between requirements and modules is listed in Table \ref{TblRT}.

\section{Module Decomposition} \label{SecMD}

Modules are decomposed according to the principle of ``information hiding'' proposed by \citet{ParnasEtAl1984}. The \emph{Secrets} field in a module decomposition is a brief statement of the design decision hidden by the module. The \emph{Services} field specifies \emph{what} the module will do without documenting \emph{how} to do it. For each module, a suggestion for the implementing software is given under the \emph{Implemented By} title. If the entry is \emph{OS}, this means that the module is provided by the operating system or by standard programming language libraries.  Also indicate if the module will be implemented specifically for the software.

Only the leaf modules in the hierarchy have to be implemented. If a dash (\emph{--}) is shown, this means that the module is not a leaf and will not have to be implemented. Whether or not this module is implemented depends on the programming language selected.

\subsection{Hardware Hiding Modules (\mref{m1})}
\begin{description}
    \item[Secrets:] The data structure and algorithm used to implement the virtual hardware.
    \item[Services:] Serves as a virtual hardware used by the rest of the system. This module provides the interface between the hardware and the software. So, the system can use it to display outputs or to accept inputs.
    \item[Implemented By:] OS
\end{description}

\subsection{Behaviour-Hiding Module}

\begin{description}
\item[Secrets:] The contents of the required behaviours.
\item[Services:] Includes programs that provide externally visible behaviour of
  the system as specified in the software requirements specification (SRS)
  documents. This module serves as a communication layer between the
  hardware-hiding module and the software decision module. The programs in this
  module will need to change if there are changes in the SRS.
\item[Implemented By:] --
\end{description}

% \begin{description}
%     \item[Secrets:] The design decision based on mathematical theorems, physical facts, or programming considerations. The secrets of this module are \emph{not} described in the SRS.
%     \item[Services:] Includes data structure and algorithms used in the system that do not provide direct interaction with the user. 
%     % Changes in these modules are more likely to be motivated by a desire to
%     % improve performance than by externally imposed changes.
%     \item[Implemented By:] --
% \end{description}

\subsubsection{ScrumBot Module (\mref{m2})}
\begin{description}
    \item[Secrets:] The format and structure of the input data.
    \item[Services:] Converts the input data into the data structure used by the input parameters module.
    \item[Implemented By:] scrumbot.py
\end{description}

\subsection{Software Decision Module}
\subsubsection{Meeting Types Module (\mref{m3})}
\begin{description}
    \item[Secrets:] Exported type (data structure)
    \item[Services:] None
    \item[Implemented By:] meetingTypes.py
\end{description}

\subsubsection{Meeting List Module (\mref{m4})}
\begin{description}
    \item[Secrets:] Data structure for a list of meetings
    \item[Services:] Provides the ability to add, remove, and output all data
    \item[Implemented By:] meetingList.py
\end{description}

\subsubsection{Meeting Module (\mref{m5})}
\begin{description}
    \item[Secrets:] Data structure of a meeting
    \item[Services:] Provides the ability to initialize, access (name, date, time, type, and description) and mutate (description only) a meeting
    \item[Implemented By:] meeting.py
\end{description}

\subsubsection{Task List Module (\mref{m6})}
\begin{description}
    \item[Secrets:] Data structure for a list of tasks
    \item[Services:] Provides the ability to add, remove, and output all data
    \item[Implemented By:] taskList.py
\end{description}

\subsubsection{Task Module (\mref{m7})}
\begin{description}
    \item[Secrets:] Data structure of a task
    \item[Services:] Provides the ability to initialize, access (deadline, details and feedback) and mutate (feedback and details) a task
    \item[Implemented By:] task.py
\end{description}

\subsubsection{Sprint Module (\mref{m8})}
\begin{description}
    \item[Secrets:] Data structure of a sprint
    \item[Services:] Provides the ability to initialize, access (all tasks) and mutate (task) a sprint
    \item[Implemented By:] sprint.py
\end{description}

\subsubsection{Project List Module (\mref{m9})}
\begin{description}
    \item[Secrets:] Data structure for a list of projects
    \item[Services:] Provides the ability to add, remove, and output all data
    \item[Implemented By:] projectList.py
\end{description}

\subsubsection{Project Module (\mref{m10})}
\begin{description}
    \item[Secrets:] Data structure for a project
    \item[Services:] Provides the ability to initialize, access (description, meetings, requirements and sprints) and mutate (description, meetings, requirements and sprints) a project
    \item[Implemented By:] project.py
\end{description}

\subsubsection{Generic Dictionary Module (\mref{m11})}
\begin{description}
    \item[Secrets:] Data structure of a dictionary
    \item[Services:] Provides basic functionality of a dictionary (add, remove, output all data)
    \item[Implemented By:] dictionary.py
\end{description}

\section{Traceability Matrix} \label{SecTM}

This section shows two traceability matrices: between the modules and the
requirements and between the modules and the anticipated changes.

% the table should use mref, the requirements should be named, use something
% like fref
\begin{table}[H]
    \centering
    \begin{tabular}{p{0.2\textwidth} p{0.6\textwidth}}
        \toprule
        \textbf{Requirements} & \textbf{Modules}\\
        \midrule
        BE1 & \mref{m2}\\
        BE2 & \mref{m2}, \mref{m9}, \mref{m10}, \mref{m11}\\
        BE3 & \mref{m2}, \mref{m9}, \mref{m11}\\
        BE4 & \mref{m2}, \mref{m8}, \mref{m9}, \mref{m10}, \mref{m11}\\
        BE5 & \mref{m2}, \mref{m8}, \mref{m9}, \mref{m10}, \mref{m11}\\
        BE6 & \mref{m2}, \mref{m8}, \mref{m9}, \mref{m10}, \mref{m11}\\
        BE7 & \mref{m2}, \mref{m8}, \mref{m9}, \mref{m10}, \mref{m11}\\
        BE8 & \mref{m2}, \mref{m9}, \mref{m10}, \mref{m11}\\
        BE9 & \mref{m2}, \mref{m9}, \mref{m10}, \mref{m11}\\
        BE10 & \mref{m2}, \mref{m9}, \mref{m10}, \mref{m11}\\
        BE11 & \mref{m2}, \mref{m8}, \mref{m9}, \mref{m10}, \mref{m11}\\
        LF1 & \mref{m2}\\
        LF2 & \mref{m2}\\
        LF3 & \mref{m2}\\
        UH1 & \mref{m2}\\
        UH2 & ~\\
        UH3 & ~\\
        UH4 & \mref{m2}\\
        P1 & ~\\
        P2 & ~\\
        P3 & ~\\
        P4 & \mref{m2}\\
        OE1 & \mref{m2}\\
        OE2 & ~\\
        OE3 & \mref{m2}\\
        OE4 & ~\\
        OE5 & ~\\
        OE6 & ~\\
        OE7 & ~\\
        OE8 & ~\\
        MS1 & ~\\
        MS2 & ~\\
        MS3 & ~\\
        MS4 & ~\\
        MS5 & ~\\
        S1 & \mref{m2}\\
        C1 & ~\\
        L1 & ~\\
        HS1 & ~\\
        \bottomrule
    \end{tabular}
    \caption{Trace Between Requirements and Modules}
    \label{TblRT}
\end{table}

\begin{table}[H]
    \centering
    \begin{tabular}{p{0.2\textwidth} p{0.6\textwidth}}
        \toprule
        \textbf{AC} & \textbf{Modules}\\
        \midrule
        \acref{acInput1} & \mref{m2}\\
        \acref{acInput2} & \mref{m2}\\
        \acref{acServer1} & \mref{m2}\\
        \acref{acServer2} & \mref{m2}\\
        \acref{acOutput} & \mref{m2}\\
        \bottomrule
    \end{tabular}
    \caption{Trace Between Anticipated Changes and Modules}
    \label{TblACT}
\end{table}

\section{Use Hierarchy Between Modules} \label{SecUse}

In this section, the uses hierarchy between modules is provided. \citet{Parnas1978} said of two programs A and B that A {\em uses} B if correct execution of B may be necessary for A to complete the task described in its specification. That is, A {\em uses} B if there exist situations in which the correct functioning of A depends upon the availability of a correct implementation of B.  Figure \ref{FigUH} illustrates the use relation between the modules. It can be seen that the graph is a directed acyclic graph (DAG). Each level of the hierarchy offers a testable and usable subset of the system, and modules in the higher level of the hierarchy are essentially simpler because they use modules from the lower levels.

\begin{figure}[H]
    \centering
    \includegraphics[width=0.7\textwidth]{"UsesHierarchy".png}
    \caption{Use hierarchy among modules}
    \label{FigUH}
\end{figure}

%\section*{References}

\bibliographystyle {plainnat}
\bibliography {MG}

\end{document}
